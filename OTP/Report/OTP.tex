\documentclass[11pt]{article}

\usepackage{listings}
\usepackage{xcolor}
\usepackage{graphicx}

% Use wide margins, but not quite so wide as fullpage.sty
\marginparwidth 0.5in 
\oddsidemargin 0.25in 
\evensidemargin 0.25in 
\marginparsep 0.25in
\topmargin 0.25in 
\textwidth 6in \textheight 8 in

% Define some colors to help read code easier
\definecolor{codegreen}{rgb}{0,0.6,0}
\definecolor{codegray}{rgb}{0.5,0.5,0.5}
\definecolor{codepurple}{rgb}{0.58,0,0.82}
\definecolor{backcolour}{rgb}{0.95,0.95,0.92}

% Format code in an easy to read method
\lstdefinestyle{mystyle}{
    backgroundcolor=\color{backcolour},   
    commentstyle=\color{codegreen},
    keywordstyle=\color{magenta},
    numberstyle=\tiny\color{codegray},
    stringstyle=\color{codepurple},
    basicstyle=\ttfamily\footnotesize,
    breakatwhitespace=true,         
    breaklines=true,                 
    captionpos=b,                    
    keepspaces=true,                 
    numbers=left,                    
    numbersep=5pt,                  
    showspaces=false,                
    showstringspaces=false,
    showtabs=false,                  
    tabsize=2
}
% Use custom style
\lstset{style=mystyle}

% That's about enough definitions

\begin{document}
\hfill\vbox{\hbox{Jaxon Haws}
		\hbox{Jonathan Schreiber}	
		\hbox{CPE 321, Section 3}	
		\hbox{Assignment: OTP}}\par

\bigskip
\centerline{\Large\bf Assignment: OTP}\par
\bigskip

\section{Task 1}

  In this task, we encrypted a string of text using XOR encryption with two 
  strings of the same length. Both strings are user provided input, which
  means this encryption is not as strong as a true one time pad which utilizes 
  a true random key. 
  This program reads each character of both inputs, converts each character 
  to its hex ascii code and performs an XOR operation on both characters. 
  It then pads the hex code with a leading zero if necessary. 

  \subsection{Code}
    \lstinputlisting[language=Python]{../task1.py}

  \subsection{Usage}
   
    {\tt\begin{tabbing}                                                                                                                                                                     
       \$ python3 task1.py\\
       Enter a string: {\it Darlin dont you go}\\
       Enter another string of the same length: {\it and cut your hair!} \\
       Encrypted string hex: {\it 250f164c0a1b54441601015259071449154e}\\
      \end{tabbing}}

\section{Task 2}

In this task, we encrypted the contents of a plaintext file using a randomly 
generated key of the same length as the contents of the file. We then wrote
corresponding ciphertext to a new file in ascii format, then decrypted the 
ciphertext using the same random key. 
  \subsection{Code}
    \lstinputlisting[language=Python]{../task2.py}
  \subsection{Usage}
    {\tt\begin{tabbing}                                                                                                                                                                     
       \$ python3 task2.py plaintext.txt encrypted.txt\\
      \end{tabbing}}

\section{Task 3}
  \subsection{Code}
    \lstinputlisting[language=Python]{../task3.py}
  \subsection{Usage}
    {\tt\begin{tabbing}                                                                                                                                                                     
      \$ pip3 install progressbar\\
      \$ python3 task3.py mustang.bmp encrypted-mustang.bmp decrypted-mustang.bmp\\
      \$ python3 task3.py cp-logo.bmp encrypted-cp-logo.bmp decrypted-cp-logo.bmp\\
      \$ diff mustang.bmp decrypted-mustang.bmp\\
      \$ diff cp-logo.bmp decrypted-cp-logo.bmp\\
    \end{tabbing}}
  \subsubsection{Plaintext Images}
    \begin{figure}[h!]
      \centering
      \includegraphics[width=4cm]{Images/mustang.jpg}
      \includegraphics[width=4cm]{Images/cp-logo.jpg}
      \caption{Plaintext Mustang and CP Logo}
      \label{fig:Plaintext_Images}
    \end{figure}

  \pagebreak

  \subsubsection{Ciphertext Images}
    \begin{figure}[!h]
      \centering
      \includegraphics[width=4cm]{Images/mustang_cipher.jpg}
      \includegraphics[width=4cm]{Images/cp_cipher.jpg}
      \caption{Plaintext Mustang and CP Logo}
      \label{fig:Ciphertext_Images}
    \end{figure}

  \subsubsection{Decrypted Plaintext Images}
    \begin{figure}[!h]
      \centering
      \includegraphics[width=4cm]{Images/mustang.jpg}
      \includegraphics[width=4cm]{Images/cp-logo.jpg}
      \caption{Decrypted Plaintext Mustang and CP Logo}
      \label{fig:Decrypted_Plaintext_Images}
    \end{figure}

\section{Task 4}
  \subsection{Code}
    \lstinputlisting[language=Python]{../task4.py}
  \subsection{Usage}
    {\tt\begin{tabbing}                                                                                                                                                                     
      \$ pip3 install progressbar\\
      \$ python3 task4.py\\
      \end{tabbing}}

  \pagebreak

    \subsubsection{Plaintext Images}
      \begin{figure}[ht!p]
        \centering
        \includegraphics[width=4cm]{Images/mustang.jpg}
        \includegraphics[width=4cm]{Images/cp-logo.jpg}
        \caption{Plaintext Mustang and CP Logo}
        \label{fig:Plaintext_Images_Task4}
      \end{figure}

    \subsubsection{Ciphertext Images}
      \begin{figure}[ht!p]
        \centering
        \includegraphics[width=4cm]{Images/mustang-2tp.jpg}
        \includegraphics[width=4cm]{Images/cp-logo-2tp.jpg}
        \caption{Ciphertext Mustang and CP Logo}
        \label{fig:Ciphertext_Images_Task4}
      \end{figure}
  
    \subsubsection{XOR Two Time Pad}
      \begin{figure}[ht!p]
        \centering
        \includegraphics[width=4cm]{Images/xor.jpg}
        \caption{XOR of Mustang Ciphertext and CP Logo Ciphertext}
        \label{fig:2_Time_Pad}
      \end{figure}

\section{Questions}

  \subsection{What is OTP? What are assumptions to achieve perfect secrecy?}
  OTP is a one-time pad that encrypts plaintext using a true random private 
  key of the same length of the plaintext.
  The encryption process is a simple XOR of the private key and plaintext.
  To achieve perfect secrecy, 3 assumptions are made: 
  \begin{enumerate}
    \item The private key was generated using a true random process. 
    \item The private key is the same length as the plaintext. 
    \item The private key is kept private between the sender and recipient.
  \end{enumerate}

  \subsection{From the results of Task 3, what do you observe? Are you able to derive any 
  useful information about from either of the encrypted images? What are the 
  causes for what you observe?}
  From Task 3, the results are images that appear to be random static.
  There are no discernable shapes, outlines, or anything that indicate what the 
  image originally was. The reason for this is that the plaintext image was
  XOR'ed with a random key, which produces random bytes.


  \subsection{From the results of Task 4, are you able to derive any useful information 
  about the original plaintexts from the resulting image? What are the causes 
  for what you observe?}
  The ciphertext images that were created with the same private key appear to 
  be random bytes with no discernable traits. 
  However, once they have been XOR'ed together, the resulting image is rather
  interesting. It is no longer a collection of random bytes, but rather a 
  mashed together image of the original 2 plaintext images with different colors. 
  The reason this happens is due to both ciphertext images using the same
  private key. This produces the same results as XOR'ing the two original 
  plaintext images together. This is verified with the following property: 
  $(A \oplus K) \oplus (B \oplus K) = A \oplus B$. 

\end{document}
